\documentclass[30p]{article}




\usepackage[german]{babel}
\usepackage[utf8]{inputenc}
\usepackage{amsmath}

\author{Renat Sakenov}
\title{Aufgabenblatt 5 }


\begin{document}

\maketitle

\section{erster Abschnitt} 
Text und so.. \\

\section{Tabelle}
\begin{tabular}[]{| p{3cm} | p{3cm} | c |}
test &  test & test\\  \hline
test &  test & test\\
test &  test & test\\

\end{tabular}


\section{Formeln}
\subsection{Pythagoras}

Der Satz des Pythagoras errechnet sich wie folgt: $a^2+b^2 = c^2$.\\ 
Daraus können wir die Länge der Hypothenuse wie folgt berechnen: $c = \sqrt{a^2+b^2}$.

\subsection{Summen}
Wir können auch eine Formel für Summen eingeben: \\
$s = \sum_{i=1}^{n} i = \frac{n * (n+1)}{2}$
   
\end{document} 
